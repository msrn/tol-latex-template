%!TEX TS-program = lualatex

\documentclass[a4paper,12pt]{article}
\usepackage{calc}
%\usepackage{lipsum}
\usepackage{tol}
\usepackage[titletoc, title]{appendix}
\usepackage{tocloft}
\usepackage{fancyvrb}

\usepackage[font={small, sf}, labelfont=bf]{caption}
\usepackage{chngcntr}
\counterwithout{figure}{section}
\addbibresource{sample.bib} %gradu.bib

%%%%%%%%%%%%%%%%%%%
%%%%TITLE PAGE%%%%%
%%%%%%%%%%%%%%%%%%%
\author{Firstname Surname}
%\email{} %Can be enabled from tol.sty
%\studentno{} %Can be enabled from tol.sty
\title{Formal Instructions for MSc and BSc Thesis \LaTeX \space template}
\institute{University Of Oulu}
\faculty{Faculty of Information Technology and Electrical Engineering}
\course{Master's Thesis}
\hypersetupdefault

%%dont hyphen
\tolerance=1
\emergencystretch=\maxdimen
\hyphenpenalty=10000
\hbadness=10000
\begin{document}

\maketitle

\setcounter{page}{2}

%%%%%%%%%%%%%%%%%%%
%%%%%ABSTRACT%%%%%%
%%%%%%%%%%%%%%%%%%%
\section*{Abstract}
\addcontentsline{toc}{section}{Abstract}
Abstract is needed to sum the master’s thesis up. The abstract is to be uploaded into Optima before the final grading of the thesis. Please find the current information about the format given in Optima.

The guide includes instructions for students. It is written keeping in mind the idea that the user may utilise it e.g. by pasting his or her text on the current text. The contents include information about formatting the text, positioning tables and figures, among other things. In addition, the use of proper literature is instructed. Even if there is no strict structure for the thesis, a recommendation is offered in this guideline.

One important guideline for the text is that do not write too short paragraphs. For instance, if there is only one sentence in a paragraph, the sentence must be really important and influential to form a paragraph of its own.

It is not possible to provide information in a guideline like this for all issues related to master’s thesis. For example, the research process, ways to acquire research material and its analysis are excluded in the guideline. On the other hand, a structure for a research plan is provided in the appendices.
\\
\\
{\sffamily\itshape Keywords}\\
first keyword, second keyword, other keywords
\\
\\
{\sffamily\itshape Supervisor}\\
Title, position First name Last name

\newpage
%%%%%%%%%%%%%%%%%%%
%%%%%FOREWORD%%%%%%
%%%%%%%%%%%%%%%%%%%
\section*{Foreword}
\addcontentsline{toc}{section}{Foreword}

The foreword is not instructed by the supervisors. In other words, the student may write in this section what she or he wants to share with readers. However, it is a custom to thank all those who have contributed to the research somehow. When acknowledging people, their affiliations are given (e.g. Professor, University Lecturer, Adjunct Professor, Mrs.) 
This guideline is based on the previous version that was written in Finnish and finalised by Dr. Lasse Harjumaa in January 2007. This version is to replace the earlier version. I want to thank all those people who have contributed to the earlier versions and this newest version, the first written in English.
Hopefully this guideline will serve both students and faculty with its instructions that include both formal and informal regulations and recommendations. In the first phase, the constructive comments are received with pleasure by raija.halonen@oulu.fi.
Oulu, January 10, 2011

Raija Halonen
Oulu, March 10, 2020

\newpage
%%%%%%%%%%%%%%%%%%%%%%%%%%%%
%%%%%TABLE OF CONTENTS%%%%%%
%%%%%%%%%%%%%%%%%%%%%%%%%%%%
\tableofcontents
\addcontentsline{toc}{section}{Contents}
\addtocontents{toc}{\protect\thispagestyle{fancy}} % to let the page number be on the right place

\newpage
%%%%%%%%%%%%%%%%%%%%%%%%%%%%
%%%%%INTRODUCTION%%%%%%%%%%%
%%%%%%%%%%%%%%%%%%%%%%%%%%%%
\section{Introduction}

In the thesis we follow the style introduced by The American Psychological Association (APA). The APA style can be found easily in the Internet and some sites provide a quick guide, too. E.g. \url{http://www.waikato.ac.nz/library/learning/g_apaguide.shtml} and \url{http://owl.english.purdue.edu/owl/resource/560/01/} are useful links.

It is important to follow given instructions. In academic theses, not only the content but also the format is important. Generally every academic publication forum requires that the publications follow their guidelines. In the theses accepted in the Department of Information Processing Science the format is APA. Currently there are several editions published from APA. The general rule is that the latest available edition is applied. Currently the newest edition is 6th. If a thesis is already in process it is not needed to transfer it into a newer edition of APA. Whichever you apply, do it consistently.

In addition to teach the students to follow given formal instructions, the guideline aims to unify and standardise the outlook of the theses made in the department. The guideline also enables the supervisors to focus on the content of the theses as the students already consider the outlook and format themselves. In this sense, it is a question of available resources for supervision and guidance.

The use of language and grammar cannot be discussed in detail in this kind of guide. However, the writing style should meet the general academic writing styles in the sense that no causeries are accepted or other lightweight texts such as jokes or rumblings. In other words, in academic theses all writing must be appropriate and reasonable. There are several guidebooks for academic writing available in the Oulu University Library, for example, and in the Internet. For those who write their thesis in Finnish there are books such as Tieteellinen kirjoittaminen \parencite{Rodriguez}. The style reference by APA (American Psychological Association, 2010) offers fruitful practical hints for writing thesis in English.

As the guideline is written according to the instructions, it enables the students to copy their text (without format) on the document and thus get their text into the right format. The format is to be used in the Bachelor’s Theses and in the Master’s Theses. In case of other theses, essays or reports it is recommended that the students inquire their teachers if the guideline is to be followed or not.

The structure of the guideline is as follows. The formal instructions for different topics are presented next. This is followed by examples of references and their use. After that the structure of theses and its writing style is discussed briefly. The guideline ends with a summary.

\newpage
\section{Formal Instructions for Text}
Every chapter starts with text. The text acts as a short introduction for the following paragraphs and chapters. The introduction informs the reader of what is to be waited next. In addition, it helps to build a contextual clearness. A proper outlook gives a positive influence of the thesis and its author. If the outlook is confused and disorganised the thesis appears negatively even if the content was reasonable. Next, the outlook and writing of theses are described.
\subsection{Title page}
The name of the thesis is written positioned 13 cm from the upper edge of the title page by the font Arial and size 20. A potential sub title is written by Arial size 16. On the right bottom of the page starting from 9 cm the name of the university, department, type of thesis, the author and date are written. If the thesis is an exercise, the student number is needed to ease crediting the performance. 
\subsection{Text and headings}
The thesis is written with Times New Roman (or similar) with single line spacing. The font is 12 points. The paragraphs are written without indentation and are separated by one empty line (12 p). Except the main headings, the headings are written by small alphabets. The main heading is in Arial font size 18 and sub headings Arial font size 14. The main headings start new pages. The heading is positioned after 48 point empty space and after the heading there is 24 point empty space. The subheadings are preceded by two empty lines (24 p) between the subheading and the following text there is empty space of one line.
On the left there is 4 cm margin to ease stapling. All other margins are 2 cm. If the amount of pages is more than 80 pages, it is recommended that the thesis is printed and stapled double-sized. In this case the margins must be mirrored. This is done e.g. in Word: Page Setup – Layout – (Headers and footers) Different Odd and Even.
According to generally adopted style, there may not be consecutive headlines. Therefore introductory text is needed after headlines before following headlines. The headlines may not remain alone without the actual text on the same page. Instead, the headlines must be tied with the following paragraph.
\subsection{Tables, figures and lists}
Tables must be numbered and named by a title. The numbering is running from the beginning of the thesis. The titles are positioned above the tables. Above the title there is space for one empty line (12 p) and below 6 points. The text must be referred according to its number. Referring by “see below” or “in the next table” is not sufficient. Table 1 sums the font styles that are used in the theses.


\begin{table}[]
\caption{The heading style in the theses.}
\label{tab:table1}
\resizebox{\textwidth}{!}{%
\begin{tabular}{|l|l|l|l|l}
\cline{1-4}
             & Font Style     & Size & NB                                                     &  \\ \cline{1-4}
Heading 0    & Arial          & 18   & \begin{tabular}[c]{@{}l@{}}No\\ numbering\end{tabular} &  \\ \cline{1-4}
Heading 1    & Arial          & 18   &                                                        &  \\ \cline{1-4}
Heading 2    & Arial          & 14   &                                                        &  \\ \cline{1-4}
Heading 3    & Arial          & 14   &                                                        &  \\ \cline{1-4}
Heading 4    & Arial, Italics & 12   & No numbering, not recommended                          &  \\ \cline{1-4}
Boxy text    & Times Roman    & 12   &                                                        &  \\ \cline{1-4}
Table title  & Arial          & 10   & Note the bolding                                       &  \\ \cline{1-4}
Figure title & Arial          & 10   & Note the bolding                                       &  \\ \cline{1-4}
\end{tabular}%
}
\end{table}

A table should not divide on two or more pages. Instead of a too long table it is recommended to build several smaller tables or to consider if there are more suitable formats to present the information such as appendices. If the contents of the table or the table as is are copied from another publication it must be cited properly. See the referencing guide.

Pictures must be numbered and named by a title. The numbering is running from the beginning of the thesis. The title is placed below the picture. Below the title is an empty space of one line (12 p). Above the picture is an empty space of 6 p. The picture is referred by its number in the text. If the picture is narrower as the body text, it is centred (see Fig. 1). 

A good hint to insert a picture is to add a table (Table – Insert – Table - 1 column, 1 row, Autofit to contents) first, and then paste the picture into the cell. Thus the picture does not float over the text and margins when editing the document. Remember to remove the table borders. Always keep the titles near the tables and pictures. 
\begin{figure}[h]
    \centering
    \includegraphics[scale=0.5]{img/example1.png}
    \caption{Example of a screen caption. In case of a long title it is indented by 1.7 cm from the left margin.}
    \label{fig:figure1}
\end{figure}

According to Finnish Laws, pictures are under legal protection. They must not be copied without permission. There are exceptions such as pictures that consist of statistical information or are produced only from numerical data. Again, proper referencing must be followed (see the APA referencing style).

Lists are used when summing issues. In academic writing the use of lists should be minimised. The lists are indented. Also lists must be discussed in the text. Similar to tables and pictures, the interpretation must be given by the author regarding lists, too.

\begin{itemize}
    \item A list is indented.
    \item The topics are talked in the body text.
    \item A chapter must not be ended with a list.
\end{itemize}

A list may also be numbered. In the next list you can see the styles utilised in the guideline. 
\begin{enumerate}
    \item ListItem
    \item ListItemLast
    \item ListItemNumbered.
\end{enumerate}
Every picture and table must be referred in the text. Before picture and after picture there must be body text. Thus, a chapter may not begin or end with a picture, table or list. 

\subsection{Emphasis and citations}
Emphasis may be used in case the topic is really significant or it must be separated from the surrounding text. For emphasising, there is a style of Emphasized. It is done by italics.
Direct citations must be made carefully. Short citations can be presented in the body text by parenthesis and including reference but longer citations are separated by making them as a paragraph and indenting it. It is important to add full references as instructed in the APA style (American Psychological Association, 2010).

\subsection{Examples of codes and formulas}
Code is presented by Courier New with size 10. The code must be explained in the body text (exception: it is axiomatic). Long lists of codes should be avoided.

\begin{verbatim}
public class B extends A {	
    public void setProp(String s) {
        this.prop = s;
    }
    private String prop;
}
\end{verbatim}

The style CodeSample is defined for the examples. Other than tables and pictures, code examples are not numbered.
Formulas are presented in the body text centred. If possible, general fonts are used. The following is a classic example of a formula.

\begin{equation}
    E = mc^2
\end{equation}

The formulas are numbered similar to tables and pictures. The number is placed on the right side. The chapter may not begin or end with a formula.
\subsubsection{Sub Sub Section}
Text Goes here

\paragraph{Paragraph}
Text Goes here

\newpage
%%%%%%%%%%%%%%%%%%%%%
%%%%%REFERENCES%%%%%%
%%%%%%%%%%%%%%%%%%%%%
%\section*{References}
%\addcontentsline{toc}{section}{References}

\printbibliography[heading=bibintoc] % If not none then another heading will be generated based on language choice
\nocite{*}

\newpage
\begin{appendices}
	
	\section{Structure for the research plan}
	
	A research plan can be reported according to the next structure. The order of the items is 
	important.
	
	\begin{refsection}
		\subsection*{Introduction }
		The topic is introduced on general level. The context of the research is described and the 
		research  problem  is  explained  and  justified.  The  problem  is  situated  in  its  larger 
		environment. Note references when needed. The researcher may reason the topic also by 
		describing his or her personal motivation. \parencite{5418289}.
		\subsection*{Research problem and research methods }
		The problem under study is explained as explicitly as possible. The research problem 
		can be divided into sub problems or presented as hypotheses. The research methods and 
		analysis are described. 
		\subsection*{Limitations}
		The planned limitations and known shortcomings are reported. The reasons for them – if 
		known – are explained from the viewpoint of the current research.
		
		\subsection*{Preliminary earlier research}
		The prior literature is presented briefly with full sentences. All required references are 
		included. Its relevance in the current research is described and limitations recognised in 
		prior research are identified if possible. 
		List of main prior literature in relation to the background theory 
		Main background references are listed in the required format (APA).
		
		\subsection*{Lähteet}
		%\nocite{*}
		%\bibliographystyle{apacite}
		%\bibliography{refs}
		\printbibliography[heading=none]
	\end{refsection}
	
	\subsection*{Timetable }
	A plan to describe the planned research related to calendar time. It is recommended that 
	the  plan  is  discussed  with  supervisor  to  ensure  enough  milestones  for  checking 
	thoroughly the status of the thesis.
	
	\subsection*{Preliminary structure of contents }
	\begin{enumerate}
		\item Introduction
		\item Glossary
		\item Prior research
		\begin{enumerate} 
			\item First
			\item Second
			\subitem Subsecond
		\end{enumerate}		
		\item Sources
	\end{enumerate}
	
\end{appendices}


\end{document}
